\chapter{Study about ElasticSearch}
\label{chap:elasticsearch}

\begin{abstract}
	This study intends to confirm that the adoption of \textit{Elasticsearch} as a database solution is feasible. Elasticsearch is a search engine system based on Apache Lucene. Its primary goal is the management of an index of documents. It does not present all the resources that an integral database solution provides. Even so, SERPRO adopted Elasticsearch as the main database solution in the architecture of the Cognitive Computing Project. The present study investigates the limitations and benefits in the use of Elasticsearch as a database.
\end{abstract}

\section{Introduction}
\label{sec:intro}

\section{Elasticsearch applications}
\label{sec:applications}
We obtained a list of practical applications after a brief consultation to the article \textit{Elasticsearch} in the Portuguese Wikipedia\cite{wiki:pt:elasticsearch} and in the English Wikipedia\cite{wiki:en:elasticsearch}.

\subsection*{Application as the main database}
\label{sec:appmaindb}

\subsection*{Application with an external database}
\label{sec:appexternaldb}
The applications in this section adopt the Elasticsearch only as a search engine, indexing documents from external databases.

\subsubsection*{Center for Open Science}
\label{sec:cos}
The Center for Open Science \nomenclature{COS}{Center for Open Science} (COS)\footnote{\url{http://cos.io}} is a non-profit technology organization dedicated to improving the alignment between scientific values and scientific practices\cite{elasticsearch:cos}.

COS adopted Elasticsearch as the primary search engine for all content on the \nomenclature{OSF}{Open Science Framework} Open Science Framework (OSF) \footnote{\url{http://osf.io}}, internal and external. All registered users and components of projects are indexed by Elasticsearch.

The case study did not mention which database stores the content to be indexed.

\section{Benefits of Elasticsearch}
\label{sec:benefits}

\section{Limitations of Elasticsearch}
\label{sec:limitations}

\section{Conclusion}
\label{sec:conclusion}

\bibliography{elasticsearch}