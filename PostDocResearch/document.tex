%% LyX 1.3 created this file.  For more info, see http://www.lyx.org/.
%% Do not edit unless you really know what you are doing.
\documentclass[
	% -- opções da classe memoir --
	12pt,				% tamanho da fonte
%	openright,			% capítulos começam em pág ímpar (insere página vazia caso preciso)
	oneside,			% para impressão em verso e anverso. Oposto a oneside
	a4paper,			% tamanho do papel. 
	% -- opções da classe abntex2 --
	%chapter=TITLE,		% títulos de capítulos convertidos em letras maiúsculas
	%section=TITLE,		% títulos de seções convertidos em letras maiúsculas
	%subsection=TITLE,	% títulos de subseções convertidos em letras maiúsculas
	%subsubsection=TITLE,% títulos de subsubseções convertidos em letras maiúsculas
	% -- opções do pacote babel --
	english
	]{abntex2}
	
% ---
% Pacotes básicos 
% ---
\usepackage{lmodern}			% Usa a fonte Latin Modern			
\usepackage[T1]{fontenc}		% Selecao de codigos de fonte.
\usepackage[utf8]{inputenc}		% Codificacao do documento (conversão automática dos acentos)
\usepackage{lastpage}			% Usado pela Ficha catalográfica
\usepackage{indentfirst}		% Indenta o primeiro parágrafo de cada seção.
%\usepackage{color}				% Controle das cores
\usepackage{graphicx}			% Inclusão de gráficos
\usepackage{microtype} 			% para melhorias de justificação
\usepackage{listings}           % para criar elementos do tipo lista            

\usepackage{hyperref}		
\hypersetup{hidelinks}
\usepackage{url}
\usepackage{breakurl}		
\usepackage{nomencl}

% ---
% Pacotes de citações
% ---
\usepackage[brazilian,hyperpageref]{backref}	 % Paginas com as citações na bibl
\usepackage[num]{abntex2cite}	% Citações padrão ABNT
\citebrackets[]


%\makeatletter
%\makeatother

\autor{José Renato Villela Dantas}


\titulo{Post Doctoral Research}


\local{Fortaleza}
% 
\data{2017}

% ---
% compila o indice
% ---
\makeindex
% ---
\makenomenclature

\begin{document}
	
% ----
% Retira espaço extra obsoleto entre as frases.
% ----
\frenchspacing

\selectlanguage{english}
% 
\imprimircapa

% \folhaderosto

% ---
% inserir lista de abreviaturas e siglas
% ---
\renewcommand{\nomname}{List of Acronymous}
\pdfbookmark[0]{\nomname}{las}
\printnomenclature
\cleardoublepage
% ---

% 
% \listoffigures
% 
% \listoftables
% 

% ---
% inserir o sumario
% ---
\pdfbookmark[0]{\contentsname}{toc}
\tableofcontents*
\cleardoublepage
% ---

% ---
% elementos textuais
% ---
\chapter{ideas and To Do Tasks}

\section{Ideas}

\section{Tasks}
\chapter{Study about ElasticSearch}
\label{chap:elasticsearch}

\begin{abstract}
	This study intends to confirm that the adoption of \textit{Elasticsearch} as a database solution is feasible. Elasticsearch is a search engine system based on Apache Lucene. Its primary goal is the management of an index of documents. It does not present all the resources that an integral database solution provides. Even so, SERPRO adopted Elasticsearch as the main database solution in the architecture of the Cognitive Computing Project. The present study investigates the limitations and benefits in the use of Elasticsearch as a database.
\end{abstract}

\section{Introduction}
\label{sec:intro}

\section{Elasticsearch applications}
\label{sec:applications}

\subsection{Application as the main database}
\label{sec:appmaindb}

\subsection{Application with an external database}
\label{sec:appexternaldb}

\section{Benefits of Elasticsearch}
\label{sec:benefits}

\section{Limitations of Elasticsearch}
\label{sec:limitations}

\section{Conclusion}
\label{sec:conclusion}
%\chapter{Web Service Invocation}
% ---

% 
% \bibliographystyle{abnt-alf}
% \bibliography{Mendeley}



\end{document}